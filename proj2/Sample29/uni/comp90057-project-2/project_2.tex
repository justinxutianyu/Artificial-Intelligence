\documentclass{article}

\usepackage{amsmath}
\usepackage{amssymb}
\usepackage{amsfonts}

\title{Advanced Theoretical Computer Science - Project 3}
\date{6th October 2015}
\author{Kai Hirsinger}

\begin{document}
  \pagenumbering{gobble}  
  \maketitle
  
  \newpage
  \pagenumbering{arabic}
  
  \section{}
    
    \paragraph{}
      Define the sets $S$ and $R$\\
      \begin{align*}
        S &= \{ x \mid x \in \mathbb{R} and 0 < x < 1 \}\\
        R &= \{ x \mid x \in \mathbb{R} and x > 0 \}\\  
      \end{align*}
      Let $f$ be a function  $f: S \to R$. The sets $S$ and $R$ have
      the same cardinality if $f$ is a mapping from $S$ to $R$. For $f$ 
      to be a mapping from $S$ and $R$, it must be the case that 
      every $r \in R$ has exactly one $s \in S$ where $f(s) = r$.\\
    
    \paragraph{Proof}
    The function $f(x) = \left(\frac{1}{x}\right) - 1$ is a mapping from
    $S \to R$. 
	
	\paragraph{}    
    Suppose this function is not a mapping from $S$ to $R$. If this
    were the case, there must exist some $s \in S$ where either:
    \begin{itemize}
      \item $f(s) \neq r$ for any $r \in R$
      \item There exists more than a single value $r \in R$ where $f(s) = r$
    \end{itemize}
    Observe that if there is some value $s \in S$ where $f(s) = 0$ the first
    item can be satisfied. To identify such a value, we set $f(s) = 0$ and 
    solve for $s$.
    \begin{align*}
      0 &= \left(\frac{1}{s}\right) - 1\\
      s &= 1 - 1\\
      s &= 0
    \end{align*}
    Substituting back into $f(s)$ gives:
    \begin{align*}
      f(0) &= \left(\frac{1}{0}\right) - 1
    \end{align*}
    Which is clearly invalid, indicating that no such value $s$ exists.
    TODO: Prove the second point.  

\end{document}